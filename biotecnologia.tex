\documentclass[11pt]{article}
\usepackage[spanish,USenglish,es-tabla]{babel}
\setlength{\parskip}{1em}
\usepackage[dvipsnames]{xcolor}
\usepackage[utf8]{inputenc}
\usepackage{amssymb,amsmath} % (librería  desímbolos)
\usepackage{graphicx}
\usepackage{multicol}
\usepackage{subfigure} % subfiguras
\usepackage[papersize={216mm,279mm},lmargin=2cm,rmargin=2cm,top=2cm,bottom=2cm]{geometry}
\usepackage{tcolorbox}
\usepackage{float} %aquete para mejorar la posición de flotantes
\usepackage{caption} % para referencias
\usepackage{ragged2e}
\usepackage{mathptmx}
\usepackage{makeidx}
\usepackage{xparse}
\usepackage{blindtext}
\usepackage{longtable,multirow,booktabs}
\usepackage{hyperref}

\definecolor{color1}{RGB}{232, 247, 250} % Color of the article title and sections
\definecolor{color2}{RGB}{0,20,20} % Color of the boxes behind the abstract and headings

\newtcolorbox{mybox}[2][]{colbacktitle=red!10!white, colback=blue!10!white,coltitle=red!70!black,
title={#2},fonttitle=\bfseries,#1}

\renewcommand*\thesection{\arabic{section}}
\usepackage[explicit]{titlesec}
\definecolor{myBlue}{HTML}{0088FF}

\titleformat{\section}[hang]{\Large\bfseries\sffamily}%
{\rlap{\color{myBlue}\rule[-6pt]{8.5cm}{1.2pt}}\colorbox{myBlue}{%
           \raisebox{0pt}[13pt][3pt]{ \makebox[20pt]{% height, width
                \fontfamily{phv}\selectfont\color{white}{\thesection}}
            }}}%
{15pt}%
{ \color{myBlue}#1
%
}
\titlespacing*{\section}{0pt}{3mm}{5mm}

\usepackage{fancyhdr}
\usepackage{lastpage}

\pagestyle{fancy}
\fancyhf{}
\renewcommand{\headrulewidth}{0.0mm}


\newtcolorbox{boxtext}[1]{colback=red!5!white,colframe=red!75!black,fonttitle=\bfseries,title=#1}

\rfoot{\colorbox{color1}{\thepage \hspace{1pt} {\bf{/}} \pageref{LastPage}}}
\cfoot{\colorbox{color1}{C.A. Salvador Jiménez R.}}
\lfoot{\colorbox{color1}{Writing Report}}


\begin{document}

\rule{\linewidth}{0.2mm}
\parbox{5cm}{
\includegraphics[scale=0.1]{https://aula.untrm.edu.pe/pluginfile.php/1/theme_moove/logo/1703941048/Sin%20t%C3%ADtulo-1.png} 
}\parbox{14cm}{ 
\centering
{\large \bf Quantum Chemitry Approach to Chemical Concepts and Atomic Properties} \\ Written Report }

\noindent {\bf{Student:}} Jiménez Rosas Carlos Alberto Salvador \hfill {\bf{Date:}} November 23, 2022
\rule{\linewidth}{0.1mm}

\begin{mybox}[detach title,before upper={\tcbtitle\quad}]{ABSTRAC:}
A series of molecules that contain a ligand and a metallic center are proposed, which we know in advance will generate high electric current flows, consequently high induced magnetic fields ($\vec{B}_{induced}$), making the calculations with DFT with and no empirical dispersion. Changes in displacement will be evaluated through $^1$H - NMR titrations, to establish the displacement-concentration relationship.  \\
\dotfill \\
{\bf{ Keywords: Chemical shift, $^1$H-NMR, Magnetic field, electric current flow.}}
\end{mybox}

\begin{multicols}{2}
    

\section{Introducción}

- Definición de OGM y su importancia en la agricultura
- Antecedentes históricos y contexto actual
- Objetivos y alcance de la monografía

\section{Fundamentos de la ingeniería genética en plantas}

- Conceptos básicos de genética y biotecnología
- Técnicas de modificación genética: PCR, ADN recombinante, etc.
- Vectores y métodos de transformación genética
\section{Aplicaciones de los OGM en plantas}
- Mejora de la productividad y resistencia a enfermedades
- Tolerancia a herbicidas y pesticidas
- Mejora de la calidad nutricional y valor agregado
- Ejemplos de cultivos modificados: maíz, soja, algodón, etc.
\section{Beneficios y ventajas de los OGM}
- Mayor eficiencia y productividad en la agricultura
- Reducción del uso de pesticidas y herbicidas
- Mejora de la seguridad alimentaria y nutricional
- Oportunidades económicas y sociales
\section{Controversias y riesgos asociados a los OGM}
- Preocupaciones sobre la seguridad alimentaria y salud humana
- Impacto ambiental y biodiversidad
- Regulación y etiquetado de productos OGM
- Debate ético y social
\section{Regulación y política de los OGM}
- Marco regulatorio internacional y nacional
- Agencias reguladoras y organismos de control
- Políticas y leyes relacionadas con los OGM
\section{Futuro y perspectivas de los OGM}
- Avances tecnológicos y innovaciones
- Desafíos y oportunidades en la agricultura sostenible
- Rol de los OGM en la seguridad alimentaria global
\section{Conclusión}
- Resumen de los principales puntos
- Reflexión sobre los beneficios y desafíos de los OGM
- Recomendaciones para futuras investigaciones y políticas

\end{multicols}

\section{Referencias}
- Lista de fuentes consultadas y citadas en la monografía
\section{Apéndices}
- Información adicional que no se incluyó en el cuerpo de la monografía (por ejemplo, tablas, gráficos, protocolos de laboratorio)

Recuerda que esta estructura es solo una guía y puede variar según tus necesidades y objetivos específicos. ¡Buena suerte con tu monografía!

\end{document}